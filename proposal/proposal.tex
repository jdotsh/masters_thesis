\documentclass[a4paper,10pt]{article}

\usepackage{lmodern}
\usepackage[T1]{fontenc}
\usepackage[utf8]{inputenc}

\usepackage[backend=bibtex]{biblatex}
\usepackage{comment}
\usepackage{graphicx}
\usepackage[pdfborder={0 0 0}]{hyperref}
\usepackage{listings}
\usepackage[usenames,dvipsnames,table]{xcolor}

\bibliography{bibliography.bib}

\definecolor{Gray}{gray}{0.5}
\definecolor{OliveGreen}{cmyk}{0.64,0,0.95,0.40}

\pdfstringdefDisableCommands{\def\citeauthor#1{#1}}

\lstset{
    language=C++,
    basicstyle=\ttfamily,
    keywordstyle=\color{OliveGreen},
    commentstyle=\color{Gray},
    captionpos=b,
    breaklines=true,
    breakatwhitespace=false,
    showspaces=false,
    showtabs=false,
    numbers=left,
}

\title{Master's Thesis Proposal \\
       KLSM: A Relaxed Concurrent Priority Queue \\
       Technical University of Vienna}
\author{Jakob Gruber, 0203440 \\
        Advisor: Prof. Dr. Scient. Jesper Larsson Tr\"aff}

\begin{document}

\maketitle

\begin{comment}
http://www.informatik.tuwien.ac.at/dekanat/abschluss-master

Der Anmeldung der Diplomarbeit ist ein Abstract beizufügen. Das Abstract muss strukturiert in
i) Problemstellung,
ii) erwartetes Resultat,
iii) methodisches Vorgehen,
iv) State-of-the art (inkl. mind. vier Literaturreferenzen) sowie
v) Bezug zum angeführten Studium
abgefasst werden.

Bsp 1: http://www.informatik.tuwien.ac.at/dekanat/Abstract1.pdf
Bsp 2: http://www.informatik.tuwien.ac.at/dekanat/Abstract2.pdf
\end{comment}

\section{Motivation \& Problem Statement}

Priority queues are abstract data structures which store a set of key/value
pairs and allow efficient access to the item with the minimal (maximal) key.
They are also a vital element in various areas of computer science such as
algorithmics (i.e. Dijkstra's shortest path algorithm) and operating system
(i.e. priority schedulers).

The recent trend towards multiprocessor computing requires new implementations
of basic data structures which are able to be used concurrently and scale well
to a large number of threads. In particular, lock-free structures promise
superior scalability by avoiding the use of blocking synchronization
primitives.

However, priority queues in particular are challenging to parallelize
efficiently since the \lstinline|delete_min| operation causes high contention
at the minimal (maximal) element.  Even though concurrent priority queues have
been extensively researched over the past decades, an ultimately solution has
not yet been reached.

A recent promising approach has been through relaxation of provided guarantees,
i.e.  allowing the priority queue to return one of the $k$ minimal items
instead of only being allowed to return the minimal item itself. The $k$-LSM is
a lock-free priority queue design which follows this approach and displays high
scalability in initial benchmarks. However, it is currently only available
integrated into the task-scheduling framework
Pheet\footnote{\url{www.pheet.org}} and thus cannot be compared directly to
other recent designs.

Within this thesis, a standalone version of the $k$-LSM priority queue will be
developed and compared extensively with state of the art concurrent priority
queues.

\section{Expected Results}

There are several goals of this thesis: first, to provide a solid
implementation of the $k$-LSM which may easily be compared against other
priority queues and/or used in practice. This implies not only that the
implementation must be efficient and correct, but it also needs to be reliable
and easy to to understand. Ideally, the scalability at high thread counts
compared to the current version will also be improved by following up on
several potential optimization ideas.

Second, to gain an in-depth understanding of the $k$-LSM's behavior in
different situations. In recent literature on concurrent priority queues, a
simple uniform throughput benchmark (in which each thread performs 50\%
insertions, 50\% deletions of uniformly random keys) has often been the main
performance evaluation tool - but is this benchmark appropriate and does it
accurately reflect a design's performance? And how do these data structures
perform on different machines and architectures? This thesis will investigate
answers to these questions and present comparisons against other current
priority queues.

And finally, to provide an extensive overview of the development of priority
queues, reaching from sequential queues through early lock-based concurrent
designs and several variations of lock-free skiplist-based queues, to recent
work which has often focused on various relaxation approaches. Special focus
will be given to the $k$-LSM implementation including all of its intricacies
such as memory management in a lock-free environment.

\section{Methodology}

Work on this thesis will proceed in several related parts:

\begin{itemize}
    \item Researching literature on priority queues, with special emphasis on
        modern, highly performant concurrent designs. What kind of designs
        are popular? What kind of performance can we expect? What are the
        main challenges faced by other designs?
    \item Thoroughly investigating the current $k$-LSM data structure. On
        the one hand, I need a thorough understanding of its implementation
        details in order to be successful in creating a reimplementation; and
        on the other, benchmarking the current implementation will provide a
        good performance baseline.
    \item Implementing a standalone version. Far from being a simple one-to-one 
        port, the standalone $k$-LSM will rather be a reimplementation from
        the basic principles of the $k$-LSM design. 
    \item Benchmarking and comparisons against other state of the art concurrent
        priority queues on multiple machines and varying benchmarks.
    \item Summarizing and reporting on previous steps.
\end{itemize}

\section{State of the Art}

\section{Relevance to Software Engineering \& Internet Computing}

The topic of this thesis firmly belongs to the areas of algorithm research
and parallel computing. Some closely related lectures are:

\begin{itemize}
    \item Advanced Multiprocessor Programming
    \item Algorithmics
    \item Algorithms and Data Structures
    \item High Performance Computing
    \item Parallel Computing
    \item Seminar in Algorithms
\end{itemize}

\nocite{*} % TODO: Remove me.
\printbibliography

\end{document}
