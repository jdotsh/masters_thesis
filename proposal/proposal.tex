\documentclass[a4paper,10pt]{article}

\usepackage{lmodern}
\usepackage[T1]{fontenc}
\usepackage[utf8]{inputenc}

\usepackage[backend=bibtex]{biblatex}
\usepackage{comment}
\usepackage{graphicx}
\usepackage[pdfborder={0 0 0}]{hyperref}
\usepackage{listings}
\usepackage[usenames,dvipsnames,table]{xcolor}

\bibliography{bibliography.bib}

\definecolor{Gray}{gray}{0.5}
\definecolor{OliveGreen}{cmyk}{0.64,0,0.95,0.40}

\pdfstringdefDisableCommands{\def\citeauthor#1{#1}}

\lstset{
    language=C++,
    basicstyle=\ttfamily,
    keywordstyle=\color{OliveGreen},
    commentstyle=\color{Gray},
    captionpos=b,
    breaklines=true,
    breakatwhitespace=false,
    showspaces=false,
    showtabs=false,
    numbers=left,
}

\title{Master Thesis Proposal \\
       KLSM: A Relaxed Concurrent Priority Queue \\
       Technical University of Vienna}
\author{Jakob Gruber, 0203440 \\
        Advisor: Prof. Dr. Scient. Jesper Larsson Tr\"aff}

\begin{document}

\maketitle

\section{Abstract}

Priority queues are abstract data structures which store a set of key/value pairs
and allow efficient access to the item with the minimal (maximal) key. Such queues are an important
element in various areas of computer science such as algorithmics (i.e. Dijkstra's shortest
path algorithm) and operating system (i.e. priority schedulers).

The recent trend towards multiprocessor computing requires new implementations of basic
data structures which are able to be used concurrently and scale well to a large number
of threads. In particular, lock-free structures promise superior scalability by avoiding
the use of blocking synchronization primitives.

Concurrent priority queues have been extensively researched over the past decades.
In this paper, we discuss three major ideas within the field: fine-grained locking
employs multiple locks to avoid a single bottleneck within the queue; SkipLists
are search structures which use randomization and therefore do not require elaborate reorganization
schemes; and relaxed data structures trade semantic guarantees for improved scalability.

\nocite{*} % TODO: Remove me.
\printbibliography

\end{document}
